\pdfbookmark[1]{Abstract}{Abstract} % Bookmark name visible in a PDF viewer

\begingroup
\let\clearpage\relax
\let\cleardoublepage\relax
\let\cleardoublepage\relax


\chapter*{Abstract} % Abstract name
\medskip


In a Ph.D. thesis, an abstract is a concise summary of the research, its objectives, methods, key findings, and conclusions. The abstract is typically found at the beginning of the thesis, and it serves as a brief overview of the entire document. Components of an abstract are:

\begin{enumerate}


\item Research Topic and Purpose: The abstract should start by stating the research topic or problem that the thesis addresses. It should also mention the research's primary objectives or goals.

\item  Methodology: A brief description of the research methods and approaches used in the study. This may include information on data collection, experiments, surveys, or any other research techniques employed.

\item Key Findings or Results: Summarize the most important and relevant findings or results of the research. These should be presented in a concise and clear manner.

\item Significance and Contribution: Explain why the research is important and what it contributes to the field. Mention any novel insights, advancements, or applications that result from the study.

\item Conclusion: Provide a summary of the conclusions drawn from the research. What can be inferred from the findings, and what are the implications for the field or for future research?

\end{enumerate}

While the abstract appears at the beginning of the thesis, it is typically written after the completion of the entire document, as it should accurately reflect the contents of the thesis.


\paragraph{Keywords} Keyword-1; Keyword-2; Keyword-3;
\vfill
\endgroup	