\chapter{Frame of Reference}
\label{chapter:theory}

In a Ph.D. thesis or any research paper, a \texttt{frame of reference}  typically refers to the theoretical or conceptual framework that serves as a foundation for your study. This framework provides a structure for understanding, interpreting, \& analyzing the research problem and the data or evidence you present. It helps readers \& researchers in your field to contextualize and make sense of your work.


\section{Considerations}
As you develop this chapter, the chapter title can be one of the following: Theory, Theoritical Background, Frame of Reference.
Here are the key aspects of a frame of reference in a Ph.D. thesis:

\begin{enumerate}
\item Theoretical Framework: This involves the theoretical perspectives, concepts, and models that you draw upon to guide your research. It helps you explain and make sense of the phenomena you're investigating.

\item Conceptual Framework: This is a more specific subset of the theoretical framework. It consists of key concepts and relationships that are directly relevant to your research. It provides a structure for analyzing and interpreting your data.

\item Related Literature: The frame of reference often includes a review of relevant literature, which demonstrates how your study is situated within the existing body of knowledge in your field. This literature review helps establish the context and importance of your research.

\item Research Questions or Hypotheses: The frame of reference should connect your theoretical and conceptual framework to your specific research questions or hypotheses. It shows how the established theories and concepts are applied to your study.

\item Justification for Framework Choice: You should explain why you selected this particular frame of reference and how it aligns with your research objectives. What makes it suitable for your study, and why is it the best fit among available alternatives?

\item Methodological Implications: Describe how your frame of reference influences the choice of research methods and data analysis techniques. Theoretical and conceptual frameworks can guide the entire research process.


\item Practical Applications: If applicable, discuss how the knowledge derived from your frame of reference can be practically applied in your field or in solving real-world problems.

\item Limitations and Critiques: Acknowledge any limitations or critiques of your chosen frame of reference. No framework is perfect, and it's essential to recognize its weaknesses.

\end{enumerate}

\section{Summary}
In essence, the frame of reference helps set the intellectual context for your research, and it's a crucial part of any academic work. It provides a roadmap for the reader to understand how you approach your research, which theories or models you rely on, and why your study is relevant and significant within your field.
 