\chapter{Result}   
\label{chapter:result}
This chapter is where you present the findings of your research complemented by a detailed account of the data, evidence, and outcomes of your study. 

\section{Considerations}
As you develop this chapter, the chapter title can be one of the following: Result, Framework, Case Analysis
Here's what should be included in the results chapter:

\begin{enumerate}

\item Presentation of Data: Display your data in a clear and organized manner. This can include tables, charts, graphs, figures, or any other appropriate visual representations to help readers understand the results.

\item Description of Data: Provide a written description or narrative of the data presented. Explain what the data represents, how it was collected, and any notable characteristics or patterns.


\item Raw Data: In some cases, you may choose to include raw data or transcripts in appendices for readers who want to delve deeper into your findings. However, this is not always necessary and should be considered carefully.

\item Visual Aids: Ensure that visual aids are labeled, appropriately titled, and properly referenced in the text. Make it clear how they relate to your analysis and conclusions.


\item Avoid Interpretation: The results chapter is primarily focused on presenting the data and findings. Avoid interpreting the results in this section; that is the role of the discussion chapter that follows.


\end{enumerate}


\section{Summary}
The results chapter should be a factual and transparent presentation of your findings, allowing readers to draw their own conclusions about the significance of the data. It is in the subsequent discussion chapter that you will provide interpretation, context, and the implications of the results, drawing connections to your research objectives and the existing body of knowledge in your field.

   





 
 


