\chapter{Conclusion}
\label{chapter:conclusion}

This  chapter serves as the final section of your research document, summarizing and synthesizing the key points, findings, and implications of your study. It offers a sense of closure and provides a clear and concise overview of the entire research project.



\section{Consideration}
Here's what should be included in the conclusion chapter:

\begin{enumerate}


\item Summary of Key Findings: Begin by summarizing the main findings and results of your research. Highlight the most significant and relevant outcomes of your study.

\item Restate the Research Problem: Remind the reader of the specific research problem or questions that your study aimed to address. Restate them clearly and concisely.

\item Review of Research Objectives: Recap the research objectives and goals you set out to achieve at the beginning of your thesis. Explain whether you have met these objectives.

\item Contributions to Knowledge: Discuss the contributions your research has made to the field. Emphasize the novel insights, advancements, or new knowledge generated by your study.

\item Theoretical and Practical Implications: Explain the theoretical and practical implications of your research. How does your work impact the theoretical understanding of the topic, and what are its real-world applications?

\item Limitations: Acknowledge any limitations or constraints in your study, and discuss how they may have influenced the results or the generalizability of your findings.

\item Reflection on Methodology: Reflect on the research methods and methodology used in your study. Discuss the strengths and weaknesses of your approach and any lessons learned for future research.

\item Future Directions: Offer suggestions for future research that could build upon your work. Identify specific areas or questions that remain unanswered or require further investigation.

\item Practical Recommendations (if applicable): If your research has practical applications, provide recommendations for policymakers, practitioners, or relevant stakeholders based on your findings.

\item Personal Reflection (optional): Some researchers choose to include a brief personal reflection on the research process, discussing their own growth, challenges, and insights gained during the study.

\item Final Thoughts and Closing Remarks: End the conclusion chapter with a compelling final thought or statement that leaves a strong impression on the reader and summarizes the significance of your research.


\end{enumerate}

\subsection{Summary}
The conclusion chapter should be concise and focused while still providing a comprehensive overview of the study's major elements. It should leave the reader with a clear understanding of the research's importance, the contributions it has made, and the potential for future research in the field.

\section{Not an Abstract}
The abstract and the conclusion chapter of a Ph.D. thesis serve different purposes and are placed at different points within the document. Here are the key differences between an abstract and the conclusion chapter:

\subsection*{Content}
 
An abstract serves as a condensed summary of the research, highlighting the key aspects of the thesis, including the research problem, objectives, methods, key findings, and conclusions. It does not provide in-depth details.

The conclusion chapter is where the researcher interprets the results, discusses their implications, summarizes the major findings, reflects on the research process, and provides recommendations for further study or practical applications.

\subsection*{Audience}
The abstract is designed to give potential readers a quick understanding of the thesis's content, allowing them to decide whether they want to read the full document. It is often used as a reference when searching for relevant research.

The conclusion chapter is intended for readers who have read the entire thesis and are looking for a thorough understanding of the research's significance and implications

\subsection*{Use}
The abstract is useful for busy researchers who want to assess the relevance of the thesis to their own work or for readers seeking a quick overview of the research without delving into the entire document.

The conclusion chapter is an integral part of the thesis, allowing the researcher to demonstrate their critical thinking and synthesis of the research. It is where the researcher has the opportunity to tie together all the threads of the study and leave the reader with a comprehensive understanding of the research's impact and future directions.

\subsection*{Last Section}
In summary, while both the abstract and the conclusion chapter provide summaries of a Ph.D. thesis, the abstract is a brief, initial overview intended for a broad audience, while the conclusion chapter is a more detailed and comprehensive analysis and reflection, serving as the closing section for those who have read the entire document.