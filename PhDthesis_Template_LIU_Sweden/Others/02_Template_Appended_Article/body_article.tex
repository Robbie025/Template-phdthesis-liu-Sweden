\section{Introduction}
In the context of a PhD thesis, \textit{appended articles} refer to a collection of research papers or articles that have been published or are ready for publication and are included as part of the thesis. 
They are also referred to as:
\begin{inparaenum}
\item thesis by publication;
\item cumulative thesis
\end{inparaenum}
%inparaenum
This format is more common in certain fields like the sciences, engineering, and some social sciences, and it is less common in humanities.

The advantage of using appended articles in a PhD thesis is that it allows researchers to showcase their work in a format that is readily accepted for publication in peer-reviewed journals, which can be beneficial for their academic and professional careers. 

\section{Final Thoughts}

Quick thoughts that I want to write down so that I don't forget.

\begin{enumerate}
\item I used Texmake for writing and setting all my articles and this PhD thesis.
\item It made it much easier to set up this work as I had my articles written with Latex from the very beginning. Otherise, I would have had to attached a scaled pdf from the original publication.
\item This thesis is the main objectives of a PhD work. Research is separate.
\item I tried to maintain one .bib file from the very beginning. citation1 \cite{ref1}, citation1 \cite{ref3}, citation:  \cite{ref2}
\item I also tried ot maintain one figure folder.
\item The structure in the thesis is only guideline and not a formulae.
\item Life is given meanig by your thoughts and actions. 

\end{enumerate}
\begin{quote}
\begin{center}
Favorite ongoing thought: Since I am not God, I use models for understanding and communicating.

\end{center}

\end{quote}
\section*{Acknowledgements}
 