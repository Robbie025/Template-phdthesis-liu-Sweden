\thispagestyle{empty}
 \begin{center}

\texttt{Link\"{o}ping Studies in Science and Technology:~Dissertation No.~2026}
%{\texttt{VARUN GOPINATH}}  
  \end{center}
\sloppy

%\begin{mdframed}[style=sidebar,frametitle={Abstract: }]  
  \begin{small}
\noindent\textbf{Abstract:}
\medskip

\noindent This thesis pertains to industrial safety in relation to human-robot collaboration. 
The aim is to enhance understanding of the nature of systems where large industrial robots collaborate with humans to complete assembly tasks. This understanding may support development and safe operations of future collaborative systems. 

Industrial robots are widely used to automate manufacturing operations across several industries. The automotive industry is the largest user of robots and have identified robot-based automation as a strategy to improve efficiency in manufacturing operations.

Recently, a class of machines referred to as \textit{collaborative robots} have been developed by robot manufacturers to support operators in assembly tasks. The use of these robots to support human workers in an industrial context are referred to as \textit{collaborative operations}.

Presently, collaborative robots have limited reach and load carrying capacity compared to standard industrial robots. 
Large/standard industrial robots are widely used for applications such as welding or painting. They can, in principle support operators in assembly tasks as well. 
 

Two laboratory demonstrators representing the final results from a series of research activities will be presented. They were developed to investigate issues related to personnel and process safety while working with large industrial robots in collaborative operations. The demonstrators were partially based on assembly workstations that are currently operational and they exemplify challenges faced by the automotive industry.

Demonstrator-based Research, a methodology for collaborative research that emphasizes development of demonstrators as a research tool, forms the rationale for carrying out research operations presented in this thesis. 
An evaluation of the laboratory demonstrators by industrial participants suggests an increased interest and confidence in collaborative operations with large robots. The demonstrators have  served as a tentative platform for participants to identify and discuss manufacturing and safety challenges in relation to their organization.

A main outcome presented in this thesis relates to specifying requirements for introducing robots in a human-populated environment. 
Introducing robotic systems in new environments requires reconsideration of the nature of the hazards particular to the domain.  An analysis of the laboratory demonstrators suggest that, in addition to hazards associated with normal functioning of the system, limitations in human cognition must be considered. These results will be exemplified and discussed in the context of situational and mode awareness. Additionally, a model of a collaborative workstation will be presented in terms of three constituents -- workspace, tasks and interaction. 

This is particularly significant considering the direction of present-day research aimed at introducing robots across various industries and working environments. In response to this trend, this thesis discusses the relevance of Interactive Research and its emphasis on \emph{joint learning} that goes on between academic researchers and industrial participants as a valuable principle for collaborative research.

  
  \end{small}
  \medskip
  
%%   \end{mdframed} 
%\noindent \myFaculty \newline
% \myDepartment \newline
% \myUni, \myLocation  


\begin{minipage}[b]{0.45\textwidth}
\includegraphics[width=6.50cm]{fig/logo}
\end{minipage}\quad~
 \begin{minipage}[b]{0.42\textwidth}
\myFaculty \newline
 \myDepartment \newline
 \myUni \newline
 SE-581 83, \myLocation \newline
\end{minipage}
